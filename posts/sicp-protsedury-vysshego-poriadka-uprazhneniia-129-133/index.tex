.. title: SICP Процедуры высшего порядка, упражнения 1.29-1.34
.. slug: sicp-protsedury-vysshego-poriadka-uprazhneniia-129-133
.. date: 2020-01-14 12:43:07 UTC+03:00
.. tags: sicp, scheme, higher_order_procedures
.. category: 
.. link: 
.. description: 
.. type: text


\chapter{Упражнение 1.29}

Правило Симпсона более точный метод вычисления интеграла чем метод приведенный ранее в книге. По правилу Симпсона, интеграл функции f от а до b приблизительно равен $\frac{h}{3}[y_0+4y_1+2y_2+...+2y_{n-2}+4y_{n-1}+2y_n]$, где h=(b-a)/n, $y_k=f(a+kh)$. Напишите процедуру вычисляющую интеграл по этому методу. Сравните результаты с предыдущим методом при n=100, n=1000.

\begin{codelisting}{scheme}
(define (sum-simpson term a next b controln k)
  (if (> a b)
      0
      (+ (* (controln k)(term a))
         (sum-simpson term (next a) next b controln (+ k 1)))))

(define (integral-simpson f a b n)
  (define (getdx start end steps)
    (/ (- end start) steps))
  (let ((dx (getdx a b n)))
    (define (controln step-number)
      (cond ((= step-number 0) 1)
            ((even? step-number) 2)
            (else 4)))

    (define (next-a x)
      (+ x dx))
  ;; ;; (define (add-dx x) (+ x (getdx dx)))
    (* (sum-simpson f a next-a b controln 0)
       (/ (getdx a b n) 3))
    ))

(integral-simpson cube 0 1 100);;19/75 0.25333333333333335
(integral-simpson cube 0 1 1000);; 751/3000 0.25033333333333335
\end{codelisting}

\chapter{Упражнение 1.30}

Процедура sum приведенная в книге генерирует линейную рекурсию. Процедуру можно переписать что бы сумма вычислялась итерациями. Покажите как это сделать.

\begin{codelisting}{scheme}
(define (sum term a next b)
  (define (iter a result)
    (if <??>
        <??>
        (iter <??> <??>)))
  (iter <??> <??>))
\end{codelisting}


\begin{codelisting}{scheme}
(define (sum term a next b)
  (define (iter a result)
    (if (> a b)
        result
        (iter (next a) (+ result (term a)))))
  (iter a 0))
(sum identity 1 inc 5) ; 15
\end{codelisting}

\chapter{Упражнение 1.31}

a. Процедура sum является лишь простым примером из огромного числа подобных абстракций, которые могут быть сформулированы через процедуры высшего порядка. Напишите аналогичную процедуру product которая возвращает произведение значений функции в точках заданного интервала. Покажите как определить факториал через данную процедуру. Также используйте product для вычисления приближения $\pi$ используя формулу $\frac{\pi}{4}=\frac{2\cdot4\cdot4\cdot6\cdot6\cdot8...}{3\cdot3\cdot5\cdot5\cdot7\cdot7...}$.

b. Если ваша процедура генерирует рекурсивный процесс, перепишите её на итерационный процесс.

\begin{codelisting}{scheme}
(define (product term a next b)
  (if (> a b)
      1
      (* (term a)
         (product term (next a) next b))))

(define (producti term a next b)
  (define (iter a result)
    (if (> a b)
        result
        (iter (next a) (* result (term a)))))
  (iter a 1))

(define (product-factorial n)
  (product identity 1 inc 5))
(product-factorial 5) ;; 120

(define (multiplicate-pi x)
  (square (/ (+ 4 x) (+ 5 x))))

(define (product-pi n)
  (define (sub-pi x)
    (* (/ x (+ 1 x)) (/ (+ x 2) (+ x 1))))
  (define (next-x x)
    (+ x 2))
  (* 4 (product sub-pi 2 next-x n)))
(product-pi 1000) ; 3.1431607055322663
\end{codelisting}

\chapter{Упражнение 1.32}

a. Покажите что процедуры sum и product оба являются частными случаями более общей нотации accumulate, которая объединяет набор терминов в более общую функцию.

b. сделайте рекурсивный и итерационный вариант.

\begin{codelisting}{scheme}
;; рекурсивный
(define (accumulate combiner null-value term a next b)
  (if (> a b)
      null-value
      (combiner (term a)
                (accumulate combiner null-value term (next a) next b))))

;; итерационный
(define (accumulate combiner null-value term a next b)
  (define (iter a result)
    (if (> a b)
      result
      (iter (next a) (combiner result (term a)))))
  (iter a null-value))

(define (product term a next b)
  (accumulate * 1 term a next b))
(define (sum term a next b)
  (accumulate + 0 term a next b))

(define (accumulate-factorial n)
  (product identity 1 inc 5))
(accumulate-factorial 5) ; 120
(define (accumulate-sum a b)
  (sum identity a inc b))
(accumulate-sum 1 5) ; 15
\end{codelisting}


\chapter{Упражнение 1.33}

Вы можете получить еще более общую версию accumulate, введя понятие фильтра по слагаемым. То есть складывать (или перемножать) только те значения которые удовлетворяют заданному условию. Результирующая абстракция filtered-accumulate все те же параметры как и accumulate, плюс дополнительным предикатом одного аргумента, описывающим фильтр. Напишите процедуру filtered-accumulate. Покажите как выразить следующие понятия используя её.

a. Сумма квадратов простых чисел в интервале от a до b.
b. Перемножение всех положительных целых меньших чем n, которые взаимно простых с n (GCD (i,n) = 1)

\begin{codelisting}{scheme}
;; рекурсивный
(define (filtered-accumulate combiner null-value term a next b filter)
  (if (> a b)
      null-value
      (combiner (if (filter a b) (term a) null-value)
                (filtered-accumulate combiner null-value term (next a) next b filter))))

;; итерационный
(define (filtered-accumulate combiner null-value term a next b filter)
  (define (iter a result)
    (if (> a b)
      result
      (iter (next a) (if (filter a b)
                         (combiner result (term a))
                         result)
            )))
  (iter a null-value))


;; cумма квадратов целых в диапазоне
;; передаем в фильтр два параметра a b что бы работало абстракция со следующим примером
(define (filter-prime? x end)
  (if (prime? x)
      1
      false))
(define (sum-square-primes a b)
  (filtered-accumulate + 0 square a inc b filter-prime?))
(sum-square-primes 1 5) ; 14

;; перемножение всех взаимно простых с n в диапазоне [m n]
(define (filter-relative-prime? x end)
  (if (= (gcd x end) 1)
      1
      false))

(define (product-relative-primes a b)
  (filtered-accumulate * 1 identity a inc b filter-relative-prime?))
(product-relative-primes 1 6); 5
\end{codelisting}

\chapter{Упражнение 1.34}

Предположим мы определили процедуру

\begin{codelisting}{scheme}
(define (f g)
  (g 2))
(f square) ; 4
(f (lambda (z) (* z (+ z 1)))) ; 6

(2 2)
\end{codelisting}

Что произойдет если вызвать (f f)?

Происходит ошибка ;The object 2 is not applicable. Функция f ожидает в качестве параметра получить функцию которую можно вызвать, она его и получает и вызывает функцию f с параметром 2 (рекурсивно второй раз), и пытается выполнить функцию (2 2), на что и ругается интерпретатор, потому что 2 не процедура.