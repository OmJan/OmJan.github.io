.. title: SICP 2.2 Иерархические данные и свойство замыкания (the closure property).
.. slug: sicp-22-ierarkhicheskie-dannye-i-svoistvo-zamykaniia-the-closure-property
.. date: 2020-01-18 15:18:01 UTC+03:00
.. tags: sicp, scheme, hierarchical_data, the_closure_property
.. category: 
.. link: 
.. description: 
.. type: text


\chapter{Упражнение 2.17}

Определите процедуру last-pair, которая возвращает список который содержит только последний элемент данного (непустого списка).


\begin{codelisting}{scheme}
(define (last-pair list1)
  (if (null? (cdr list1))
      list1
      (last-pair (cdr list1))))
\end{codelisting}

\chapter{Упражнение 2.18}

Определите процедуру reverse, которая возвращает список элементы которого записаны в обратном порядке.


\begin{codelisting}{scheme}
(define (reverse list1)
  (define (iter l result)
    (if (null? l)
        result
        (iter (cdr l) (cons (car l) result))))
  (iter list1 nil))
\end{codelisting}

\chapter{Упражнение 2.19}

Рассмотрим программу подсчета количества изменений из раздела 1.2.2. Было бы хорошо легко изменить валюту. Когда программа написана валюта учитывается в методах first-denomination и count-change.

\begin{codelisting}{scheme}
;; исходная программа
(define (count-change amount)
  (cc amount 5))
(define (cc amount kinds-of-coins)
  (cond ((= amount 0) 1)
        ((or (< amount 0) (= kinds-of-coins 0)) 0)
        (else (+ (cc amount
                     (- kinds-of-coins 1))
                 (cc (- amount
                        (first-denomination kinds-of-coins))
                     kinds-of-coins)))))
(define (first-denomination kinds-of-coins)
  (cond ((= kinds-of-coins 1) 1)
        ((= kinds-of-coins 2) 5)
        ((= kinds-of-coins 3) 10)
        ((= kinds-of-coins 4) 25)
        ((= kinds-of-coins 5) 50)))

(cc 100 5)
\end{codelisting}

Было бы лучше иметь список монет которые могут быть использованы для размена.

\begin{codelisting}{scheme}
(define us-coins (list 50 25 10 5 1))
(define uk-coins (list 100 50 20 10 5 2 1 0.5))
\end{codelisting}

Мы хотим переписать программу cc что бы её второй аргумент представлял собой список значений монет, вместо целого числа. Тогда у нас будут списки для каждого вида валют.

\begin{codelisting}{scheme}
(cc 100 us-coins)
292
\end{codelisting}

Для этого потребуется несколько изменить программу cc. Доступ ко второму элементу будет осуществляться по другому.

\begin{codelisting}{scheme}
(define (cc amount coin-values)
  (cond ((= amount 0) 1)
        ((or (< amount 0) (no-more? coin-values)) 0)
        (else
         (+ (cc amount
                (except-first-denomination coin-values))
            (cc (- amount
                   (first-denomination coin-values))
                coin-values)))))
\end{codelisting}

Определите процедуры first-denomination, except-first-denomination, no-more. Влияет ли порядок в списке на итоговое число?

\begin{codelisting}{scheme}
(define us-coins (list 1 25 10 5 50))
(define uk-coins (list 100 50 20 10 5 2 1 0.5))
(define (except-first-denomination list1)
  (cdr list1))
(define (first-denomination list1)
  (car list1))
(define (no-more? list1)
  (if (null? list1)
      list1
      false))
(define (cc amount coin-values)
  (cond ((= amount 0) 1)
        ((or (< amount 0) (no-more? coin-values)) 0)
        (else
         (+ (cc amount
                (except-first-denomination coin-values))
            (cc (- amount
                   (first-denomination coin-values))
                coin-values)))))
(cc 100 us-coins)
\end{codelisting}

Порядок не влияет, так как рекурсивно обходятся все варианты изменения.


\chapter{Упражнение 2.20}

Используя нотацию dotted-tail, напишите процедуру same-parity которая принимает одно или более целых чисел и возвращает список всех аргументов которые имеют такое-же четное-нечетное соотношение как первый аргумент.

\begin{codelisting}{scheme}
(define (same-parity x . l)
  (define (filter list1 same?)
    (if (null? list1)
        list1
        (if (same? (car list1))
            (cons (car list1) (filter (cdr list1) same?))
            (filter (cdr list1) same?)
            )))
  (define same-even-odd (lambda (i) (if (or (and (not (even? x)) (not (even? i))) (and (even? x) (even? i))) 1 false)))
  (cons x (filter l same-even-odd)))

(same-parity 2 3 4 5 6 7 8)
\end{codelisting}

\chapter{Упражнение 2.21}

Процедура square-list принимает список аргументов и возвращает список квадратов.

\begin{codelisting}{scheme}
(square-list (list 1 2 3 4))
(1 4 9 16)
\end{codelisting}

Приведены два реализации square-list. Дополните их.

\begin{codelisting}{scheme}
(define (square-list items)
  (if (null? items)
      nil
      (cons (square (car items)) (square-list (cdr items)))))


(define (square-list items)
  (map (lambda (x) (square x)) items))
(square-list (list 1 2 3))
\end{codelisting}


\chapter{Упражнение 2.22}

Луис пытается переписать первую реализацию square-list (2.21) что бы она генерировала итеративный процесс.

\begin{codelisting}{scheme}
(define (square-list items)
  (define (iter things answer)
    (if (null? things)
        answer
        (iter (cdr things) 
              (cons (square (car things))
                    answer))))
  (iter items nil))
\end{codelisting}

К сожалению список получается в обратном порядке, почему?

(list 1 2 3) nil
iter (list 2 3) (cons 1 nil)
iter (list 3) (cons 4 (cons 1 nil))
iter () (cons 9 (cons 4 (cons 1 nil)))
return (cons 9 (cons 4 (cons 1 nil)))

Луис пытается починить баг переставляя аргументы в cons

\begin{codelisting}{scheme}
(define (square-list items)
  (define (iter things answer)
    (if (null? things)
        answer
        (iter (cdr things)
              (cons answer
                    (square (car things))))))
  (iter items nil))
\end{codelisting}

Это не работает получается (cons (cons (cons nil 1) 4) 9)


\chapter{Упражнение 2.23}

Процедура for-each похожа на map, только вместо того чтобы возвращать измененный список она просто применяет переданную функцию к каждому элементу. Используется что бы выполнить действия над списком, например печать.

\begin{codelisting}{scheme}
(for-each (lambda (x) (newline) (display x))
          (list 57 321 88))
57
321
88
\end{codelisting}

Возвращаемое значение может быть любым например true. Дайте определение процедуры for-each.

\begin{codelisting}{scheme}
(define (for-each proc items)
  (if (not (null? items))
      (proc (car items)))
  (if (not (null? items))
      (for-each proc (cdr items)))
  true)
\end{codelisting}


\chapter{Упражнение 2.24}

Предположим мы оцениваем выражение (list 1 (list 2 (list 3 4))). Что даст вывод интерпретатора. Изобразите box-and-pointer и древовидную структуры.


\begin{codelisting}{scheme}
(1 (2 (3 4)))

(list 1 (list 2 (list 3 4)))
     /      \
    /        \
   1   (list 2 (list 3 4))
            /        \
           /          \
          2         (list 3 4)
                     /      \
                    /        \
                   3          4

(cons 1  (cons (list 2 (...)) nil)
     [.][.]- - - - - - - >[.][/]
      |                    |
      |                    |
      v                    v
      1                   [.][.]- ->[.][/]
                           |         |
                           |         |
                           v         v
                           2     (cons 3 (cons 4 nil)
                                    [.][.]- ->[.][/]
                                     |         |
                                     |         |
                                     v         v
                                     3         4
\end{codelisting}

\chapter{Упражнение 2.25}

Приведите комюинации cars и cdrs которые выбирую 7 из каждого списка.

\begin{codelisting}{scheme}
;; (1 3 (5 7) 9)
(car (cdr (car (cdr (cdr (list 1 3 (list 5 7) 9))))))
;; ((7))
(car (car (list (list 7))))
;; (1 (2 (3 (4 (5 (6 7))))))
(define list3 (list 1 (list 2 (list 3 (list 4 (list 5 (list 6 7)))))))
(car (cdr (car (cdr (car (cdr (car (cdr (car (cdr (car (cdr list3))))))))))))
\end{codelisting}

\chapter{Упражнение 2.26}

Допустим определили два списка. Какой будет результат выражений.

\begin{codelisting}{scheme}
(define x (list 1 2 3))
(define y (list 4 5 6))

(append x y)
;; (1 2 3 4 5 6)
(cons x y)
;; ((1 2 3) 4 5 6)
(list x y)
;; (cons x (cons y nil)
;; ((1 2 4) (4 5 6))
\end{codelisting}

то есть list создаст список списков, в то время как cons запишет в car первый список, а в cdr указатель на y, что приведет к разному результату с list x y. cons просто добавляет элемент в начало списка y.

\chapter{Упражнение 2.27}

Измените процедуру reverse что бы она умела разворачивать вложенные списки.
\begin{codelisting}{scheme}
(define x (list (list 1 2) (list 3 4)))

(define (deep-reverse list1)
  (define (iter l result)
    (if (null? l)
        result
        (iter (cdr l) (cons (deep-reverse (car l)) result))))
  (if (pair? list1)
      (iter list1 nil)
      list1))
\end{codelisting}

тут просто рекурсивно вызываем deep-reverse и если list1 это не список просто возвращаем значение.

\chapter{Упражнение 2.28}

Напишите процедуру fringe, которая принимает дерево в виде списка и возвращает список всех листьев в порядке слева на право.

Например:
\begin{codelisting}{scheme}
(define x (list (list 1 2) (list 3 4)))

(fringe x)
(1 2 3 4)

(fringe (list x x))
(1 2 3 4 1 2 3 4)
\end{codelisting}


\begin{codelisting}{scheme}
define (fringe x)
  (if (null? x)
      x
      (if (pair? (car x))
          (append (fringe (car x)) (fringe (cdr x)))
          x)))
\end{codelisting}

\chapter{Упражнение 2.29}

Бинарный ползунок состоит из двух веток, левой и правой. Мы можем представить бинарный ползунок используя объединение из двух веток (к примеру используя list).

\begin{codelisting}{scheme}
(define (make-mobile left right)
  (list left right))
\end{codelisting}

 Каждая ветка это стержень определенной длинны + вес либо еще один бинарный ползунок.

\begin{codelisting}{scheme}
(define (make-branch length structure)
  (list length structure))
\end{codelisting}

a. Напишите селекторы left-branch и right-branch, которые возвращабт ветки + branch-length и branch-structure, которые возвращают компоненты ветки.
b. Используя селекторы напишите процедуру total-weight, которая возвращает суммарный вес.
с. Ползунок сбалансирован, когда крутящий момент верхней левой ветки равен верхней правой ветке (то есть длинна стержня умножить на вес свисающий с этого стержня). Напишите процедуру которая тестирует сбалансировал ли ползунок.
d. Предположим мы поменяли представление так что конструктор стал таким

\begin{codelisting}{scheme}
(define (make-mobile left right)
  (cons left right))
(define (make-branch length structure)
  (cons length structure))
\end{codelisting}

Как много изменений нужно внести в программу после этого?

\begin{codelisting}{scheme}
(define (make-mobile left right)
  (list left right))

(define (make-branch length structure)
  (list length structure))
(define (left-branch m)
  (car m))
(define (right-branch m)
  (car (cdr m)))
(define (branch-length b)
  (car b))
(define (branch-structure b)
  (car (cdr b)))


(define l0 (make-branch 3 10))
(define r0 (make-branch 3 10))
(define m0 (make-mobile l0 r0))

(define l1 (make-branch 10 10))
(define r1 (make-branch 5 m0))
(define m1 (make-mobile l1 r1))

;; общий вес
(define (structure-weight s)
  (if (pair? s)
      (total-weight s)
      s))
(define (branch-weight b)
  (structure-weight (branch-structure b)))

(define (total-weight m)
  (+ (branch-weight (left-branch m)) (branch-weight (right-branch m))))

(total-weight m1)

;; балансировка
(define (balanced? m)
  (= (* (branch-weight (left-branch m)) (branch-length (left-branch m)))
     (* (branch-weight (right-branch m)) (branch-length (right-branch m)))))

(balanced? m1)
(balanced? m0)
(define (make-mobile left right)
  (cons left right))
(define (make-branch length structure)
  (cons length structure))

(define (left-branch m)
  (car m))
(left-branch m0)
(define (right-branch m) ;; поменял метод
  (cdr m))
(right-branch m0)
(define (branch-length b)
  (car b))
(define (branch-structure b) ;; поменял метод
  (cdr b))
\end{codelisting}

поменяли два селектора.


\chapter{Упражнение 2.30}

Напишите процедуру square-tree без и с использованием map.

\begin{codelisting}{scheme}
(define (square-tree tree)
  (cond ((null? tree) nil)
        ((not (pair? tree)) (square tree))
        (else (cons (square-tree (car tree))
                    (square-tree (cdr tree))))))

(define (square-tree tree)
  (map (lambda (sub-tree)
         (if (pair? sub-tree)
             (square-tree sub-tree)
             (square sub-tree)))
       tree))
\end{codelisting}

\chapter{Упражнение 2.31}

Абстрагируйте ваш ответ к 2.30 что бы можно было записывать процедуру square-tree через tree-map.

\begin{codelisting}{scheme}
(define (square-tree tree) (tree-map square tree))


(define (tree-map proc tree)
  (map (lambda (sub-tree)
         (if (pair? sub-tree)
             (tree-map proc sub-tree)
             (proc sub-tree)))
       tree))
\end{codelisting}

\chapter{Упражнение 2.32}

Мы можем представить set как список уникальных элементов, а так же set всех подмножеств как список списков. К примеру set (1 2 3), тогда сет всех подмножеств будет (() (3) (2) (2 3) (1) (1 3) (1 2) (1 2 3)). Допишите функцию генерирующую список всех подмножеств сета и дайте пояснение как она работает.

\begin{codelisting}{scheme}
(define (subsets s)
  (if (null? s)
      (list nil)
      (let ((rest (subsets (cdr s))))
        (append rest (map <??> rest)))))
\end{codelisting}

\begin{codelisting}{scheme}
(define (subsets s)
  (if (null? s)
      (list nil)
      (let ((rest (subsets (cdr s))))
        (append rest (map (lambda (set) (cons (car s) set)) rest)))))

(subsets (list 1 2 3))
;Value: (() (3) (2) (2 3) (1) (1 3) (1 2) (1 2 3))
\end{codelisting}

выбираем список без первого элемента, генерируем для него все подмножества. добавляем к полученным подмножествам такое же подмножество но везде первым элементом идет первый элемент.
на последней итерации у нас получается пустое список, он возвращает () - нулевое подмножество. в итоге мы перебрали все возможные варианты.