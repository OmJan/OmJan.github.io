.. title: SICP Упражнения 1.14, 1.15, 1.16, 1.17.
.. slug: sicp-uprazhnenie-114
.. date: 2020-01-11 14:54:25 UTC+03:00
.. tags: sicp, scheme, lisp
.. category: 
.. link: 
.. description: 
.. type: text

\chapter{Упражнение 1.14}

\section{Изобразите дерево иллюстрирующее процесс процедуры count-charge из секции 1.22 для 11 центов.}

\begin{codelisting}{scheme}
;; Example: Counting change
;; Сколькими способами мы можем внесни изменение в сумму
;; если у нас есть монеты 50 25 10 5 и 1 центов.
;; Есть простое решение через рекурсивную процедуру
;; если a = 0 1 способ внести изменение
;; если a < 0 0 способов
;; если n = 0, 0 способов внести изменение
(define (count-change amount)
  (cc amount 5))
(define (first-denomination kinds-of-coins)
  (cond ((= kinds-of-coins 1) 1)
        ((= kinds-of-coins 2) 5)
        ((= kinds-of-coins 3) 10)
        ((= kinds-of-coins 4) 25)
        ((= kinds-of-coins 5) 50)))

(define (cc amount kinds-of-coins)
  (cond ((= amount 0) 1)
        ((or (< amount 0) (= kinds-of-coins 0)) 0)
        (else (+ (cc amount
                     (- kinds-of-coins 1)) ;; изменение берем разные монетки
                 (cc (- amount
                        (first-denomination kinds-of-coins)) ;; изменяем сумму на значение монетки
                     kinds-of-coins)))))
\end{codelisting}

Иттерируя можно себе представить что получится что-то вроде.
\begin{codelisting}{scheme}
                            (cc 11 5)
                     (cc 11 4)     (cc -39 5)
                (cc 11 3) (cc -14 4)      0
           (cc 11 2) (cc 1 3) | 0
      (cc 11 1) (cc 6 2) | (cc 1 2) (cc -9 3)
 (cc 11 0) (cc 10 1) | (cc 6 1) (cc 1 1) | (cc 1 1) (cc -4 2) | 0
    0 | (cc 10 0) (cc 9 0) | (cc 6 0) (cc 5 1) | (cc 1 0) (cc 0 1) | (cc 1 0) (cc 0 1) | 0
          0 | 0 | 0 | (cc 5 0) (cc 4 1) | 0 | 0 | 0 | 0
          0 | 0 | (cc 4 0) (cc 3 1)
          0 | (cc 3 0) (cc 2 1)
          0 | (cc 2 0) (cc 1 1)
          0 | (cc 1 0) (cc 0 1)
          0 | 0
\end{codelisting}

\section{Каков порядок роста ($\theta(n)$) по памяти и по количеству шагов. Где n - сумма.}

По количеству шагов, давайте представим что у нас есть всего одна монетка для изменений. Тогда.

\begin{codelisting}{scheme}
        (cc amount 1)
  (cc amount 0)  (cc (amount - 1) 1)
0                0     (cc (amount - 2) 1)
\end{codelisting}

Видим что у нас левая часть всегда вырождается в 0. А порядок роста в право это просто $\theta(n)$

Теперь посмотрим что будет происходить с двумя монетками для изменения.

\begin{codelisting}{scheme}
        (cc amount 2)
  (cc amount 1)  (cc (amount - 1) 2)
\end{codelisting}

Видим что в левой части у нас в точности то что при одной монетке. А в провой части у нас каждый раз будет создаваться новая левая для одной монетки и еще одна правая. Порядок роста у нас получится $\theta(n^2)$.

И так при каждом добавлении еще одного номинала разменной монетки, порядок роста будет увеличиваться на n.

У нас в задаче 5 разменных монет, поэтому для нашей задачи порядок роста $\theta(n^5)$.

По памяти $\theta(n)$ (порядок максимального разветвления дерева).

\chapter{Упражнение 1.15}

Синус угла (указанный в радианах) может быть вычислен используя приближение что $\sin(r) = r$ если $r$ малое значение. Для итеративного уменьшения угла будем использовать тригонометрическое тождество $\sin{r} = 3\sin{r/3} - 4sin^3{r/3}$.

\begin{codelisting}{scheme}
(define (cube x) (* x x x))
(define (p x) (- (* 3 x) (* 4 (cube x))))

(define (sine angle)
   (if (not (> (abs angle) 0.1))
       angle
       (p (sine (/ angle 3.0)))))
\end{codelisting}

\section{Сколько раз выполнится процедура p применительно к sin 12.15?}

Будем делить угол на 3 пока, он не станет меньше чем 0.1. 12.15/3/3/3/3/3 = 0.049, значит 5 раз.

А если записать в общем виде это будет $\log_3{x} + a$, где а константа.

\section{Каков порядок роста этой функции.}

Порядок роста по количеству итераций равен $\theta(log{n})$.
По месту так же.

\chapter{Упражнение 1.16}

Написать итеративный вариант быстрой экспоненты с порядком роста $\theta(log{n})$

Рекурсивный вариант:
\begin{codelisting}{scheme}
(define (even? n)
  (= (remainder n 2) 0))
(even? 2)
(even? 7)

(define (fast-expt b n)
  (cond ((= n 0) 1)
        ((even? n) (square (fast-expt b (/ n 2))))
        (else (* b (fast-expt b (- n 1))))))
(fast-expt 5 10)
\end{codelisting}

Итеративный вариант:
\begin{codelisting}{scheme}
(define (even? n)
  (= (remainder n 2) 0))

(define (expt b n)
  (define (expt-iter a count)
    (cond ((= count 1) a)
          ((= count 0) a)
          ((>= count (- n 1)) (expt-iter (square b) (/ count 2)))
          (else (expt-iter (square a) (/ count 2)))))
  (if (even? n)
      (expt-iter 1 n)
      (* b (expt-iter 1 (- n 1)))))
(expt 3 150)
\end{codelisting}


\chapter{Упражнение 1.17}

Допустим у нас нет процедуры умножения, написать процедуру использую только сложение и предопределенные процедуры halve и double, с порядком роста $\theta(log{n})$

\begin{codelisting}{scheme}
(define (double x)
  (* x 2))
(double 2)

(define (halve x)
  (/ x 2))
(halve 4)

(define (even? n)
  (= (remainder n 2) 0))
(even? 6)

(define (fast? a b)
  (cond ((or (= b 0) (= a 0)) 0)
        ((= b 1) a)
        ((even? b) (double (fast? a (halve b))))
        (else (+ a (double (fast? a (halve (- b 1))))))))
(fast? 10 0)
(fast? 0 10)
(fast? 10 10)
(fast? 10 11)
\end{codelisting}

\chapter{Упражнение 1.18}

Используя результаты 1.16, 1.17 разработайте итеративную процедуру умножения двух чисел в терминах ``удвоить, взять половину, добавить'', процедура должна выполняться с порядком роста $\theta(log{n})$.

\begin{codelisting}{scheme}
(define (fast? a b)
  (define (fast-iter asum bsum count)
    (cond ((= bsum 0) asum)
          ((= bsum 1) asum)
          ((even? bsum) (fast-iter (double asum) (halve bsum)))
          (else (fast-iter (+ asum asum) (- bsum 1)))))
  (cond (or (= b 0) (= a 0) 0)
        ((= b 1) a)
        (else (fast-iter a b))))
(fast? 16 18)
\end{codelisting}


\chapter{Упражнение 1.19}

Доработать вычисление функции Фибоначи, процедура должна выполняться с порядком роста $\theta(log{n})$.

В последнем итеративном подсчете Фибоначи мы перезаписывали a <- a + b и b <- a, назовем это трансформацией T. Заметим что применение T n раз даст пару Fib(n + 1) и Fib(n), иными словами числа генерируются применением T^n.
Теперь рассмотрим частный случай T p = 0 q = 1 в преобразовании Tpq, где Tpq преобразует пару (a, b) в a <- bq + aq + ap и b <- bp + aq.
Покажите что если применить такую трансформацию Tpq дважды, результат будет тем же как и от единственной трансформации Tp'q' такой же формы, для которой p' и q' выраженны через p и q.
Это даст нам возможность ускорить процедуру, как мы делали в fast-exp. Собрав все это вместе закончите процедуру приведенную ниже.

Выражаем p' и q' через p и q.

\begin{proposition}
a1 b1
a2 = b1q + a1q + a1p
b2 = b1p + a1q


a3 = b2q + a2q + a2p
b3 = b2p + a2q


a3 = (b1p + a1q)q + (b1q + a1q + a1p)q + (b1q + a1q + a1p)p
b3 = (b1p + a1q)p + (b1q + a1q + a1p)q

a3 = b1q' + a1q' + a1p' (1)
b3 = b1p' + a1q' (2)

a3 = b1pq + a1qq + b1qq + a1qq + a1pq + b1pq + a1pq + a1pp
b3 = b1pp + a1pq + b1qq + a1qq + a1pq

a3 = b1(2pq + qq) + a1(2qq + 2pq + pp) (3)
b3 = b1(pp + qq) + a1(2pq + qq) (4)

a3 = b1(2pq + qq) + a1(2pq + qq) + a1(qq + pp)


q' = 2pq + qq

p' = pp + qq
\end{proposition}

\begin{codelisting}{scheme}
(define (fib n)
  (fib-iter 1 0 0 1 n))
(define (fib-iter a b p q count)
  (cond ((= count 0) b)
        ((even? count)
         (fib-iter a
                   b
                   (+ (square p) (square q))      ; compute p'
                   (+ (* 2 p q) (square q))      ; compute q'
                   (/ count 2)))
        (else (fib-iter (+ (* b q) (* a q) (* a p))
                        (+ (* b p) (* a q))
                        p
                        q
                        (- count 1)))))
\end{codelisting}