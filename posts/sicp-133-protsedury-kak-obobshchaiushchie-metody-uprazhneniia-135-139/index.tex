.. title: SICP 1.33 Процедуры как обобщающие методы, упражнения 1.35-1.39
.. slug: sicp-133-protsedury-kak-obobshchaiushchie-metody-uprazhneniia-135-139
.. date: 2020-01-15 01:36:40 UTC+03:00
.. tags: sicp, scheme, procedures_as_general_methods
.. category: 
.. link: 
.. description: 
.. type: text


\chapter{Упражнение 1.35}

Показать что золотое сечение $\phi$ это фиксированная точка трансформации $x\mapsto1+1/x$ и используя этот факт, вычислить $\phi$ через fixed-point.

По определению фиксированной точки $\phi=1+1/\phi$, где $\phi=(1+\sqrt{5})/2$ в общем очевидно что $\phi$ фиксированная точка на таком отображении.


\begin{codelisting}{scheme}
(define (phi x)
  (fixed-point (lambda (y) (+ 1 (/ 1 y)))
  1.0))

(phi 1) ;Value: 1.6180327868852458
\end{codelisting}

\chapter{Упражнение 1.36}

Измените процедуру fixed-point чтобы она печатала последовательность генерируемых приближений используя newline и display из упражнения 1.22. Затем найдите решение для $x^x=1000$ найдя фиксированную точку для $x\mapsto\log(1000)/\log(x)$. Сравните количество шагов с использованием усреднения и без него.

\begin{codelisting}{scheme}
(define (fixed-point f first-guess)
  (define (close-enough? v1 v2)
    (< (abs (- v1 v2)) tolerance))
  (define (try guess)
    (newline)
    (display guess)
    (let ((next (f guess)))
      (if (close-enough? guess next)
          next
          (try next))))
  (try first-guess))
\end{codelisting}

\begin{codelisting}{scheme}
(define (xx1000)
  (fixed-point (lambda (y) (/ (log 1000) (log y)))
               2.0))
(xx1000)
\end{codelisting}

2.
9.965784284662087
3.004472209841214
6.279195757507157
3.759850702401539
5.215843784925895
4.182207192401397
4.8277650983445906
4.387593384662677
4.671250085763899
4.481403616895052
4.6053657460929
4.5230849678718865
4.577114682047341
4.541382480151454
4.564903245230833
4.549372679303342
4.559606491913287
4.552853875788271
4.557305529748263
4.554369064436181
4.556305311532999
4.555028263573554
4.555870396702851
4.555315001192079
4.5556812635433275
4.555439715736846
4.555599009998291
4.555493957531389
4.555563237292884
4.555517548417651
4.555547679306398
4.555527808516254
4.555540912917957
;Value: 4.555532270803653

35 раз.

\begin{codelisting}{scheme}
(define (xx1000)
  (fixed-point (lambda (y) (average y (/ (log 1000) (log y))))
               2.0))
(xx1000)
\end{codelisting}

2.
5.9828921423310435
4.922168721308343
4.628224318195455
4.568346513136242
4.5577305909237005
4.555909809045131
4.555599411610624
4.5555465521473675
;Value: 4.555537551999825

10 раз.

\chapter{Упражнение 1.37}

a. Бесконечная непрерывная дробь, выражение вида $f=\frac{N_1}{D_1+\frac{N_2}{D_2+\frac{N_3}{D_3+...}}}$. В качестве примера можно показать что если для всех i $N_i=1$ $D_i=1$, то дробь будут равна $1/\phi$, где $\phi$ золотое сечение рассмотренное в упражнении 1.22. Один из способов приближенного вычисления такой дроби это отсечь дальнейшее деление при каком-то i=k. Такое усечение называется k-выраженная конечная непрерывная дробь. $\frac{N_1}{D_1+\frac{N_2}{...+\frac{N_k}{D_k}}}$.
Предположим что n и d процедуры одного аргумента (в терминах индекса i), которые возвращают $N_i$ и $D_i$ в терминологии непрерывной дроби. Определите процедуру cont-frac так что выполнение (cont-frac n d k) вычисляет значение дроби усеченной по k. Проверьте вашу процедуру на приближенное вычисление $1/\phi$ используя код

\begin{codelisting}{scheme}
(cont-frac (lambda (i) 1.0)
           (lambda (i) 1.0)
           k)
\end{codelisting}

b. Напишите рекурсивную и итеративную реализацию процедуры cont-frac.

\begin{codelisting}{scheme}
(define (cont-frac n d k)
  (define (revert-index n d i)
    (if (> i k)
        (/ (n i) (+ (d i) 0.0))
        (/ (n i) (+ (d i) (revert-index n d (+ i 1))))))
  (revert-index n d 0))

(cont-frac (lambda (i) 1.0)
           (lambda (i) 1.0)
           100) ;Value: .6180339887498948


(define (cont-frac n d k)
  (define (iter i result)
    (newline)
    (display result)
    (if (= i 0)
        (/ (n i) (+ (d i) result))
        (iter (- i 1) (/ (n i) (+ (d i) result)))))
  (iter k 0.0))

(cont-frac (lambda (i) 1.0)
           (lambda (i) 1.0)
           100) ;Value: .6180339887498948
\end{codelisting}

\chapter{Упражнение 1.38}

В 1737 швейцарский математик Леонард Эйлер опубликовал мемуары De Fractionibus Continuis, которые содержали запись непрерывной дроби для e-2, где e - основание натурального логарифма. В этой дроби все Ni = 1, а Di соответственно 1, 2, 1, 1, 4, 1, 1, 6, 1, 1, 8, ... Напишите процедуру для приближенного вычисления e.

\begin{codelisting}{scheme}
(define (e-approximation k)
  (+ 2 (cont-frac (lambda (i) 1.0)
                  (lambda (i) (cond ((= i 0) 1.0)
                               ((= i 1) 2.0)
                               ((<= i 3) 1.0)
                               ((= (remainder (- i 1) 3) 0) (* 2 (/ (- i 1) 3)))
                               (else 1.0)))
                  k)))

(e-approximation 100) ;Value: 2.7208245385621153
\end{codelisting}


\chapter{Упражнение 1.39}

Бесконечная дробь для тангенса была опубликована в 1970 немецким математиком Ламбертом. $\tan{x}=\frac{x}{1-\frac{x^2}{3-\frac{x^2}{5-...}}}$. Напишите процедуру для приближенного вычисления тангенса.

\begin{codelisting}{scheme}
(define (cont-frac-params n d k x)
  (define (iter i result)
    (newline)
    (display result)
    (if (= i 0)
        (/ (n i x) (- (d i x) result))
        (iter (- i 1) (/ (n i x) (- (d i x) result)))))
  (iter k 0.0))


(define (tan-cf x k)
  (cont-frac-params (lambda (i j) (cond ((= i 0) j)
                          (else (square j))))
           (lambda (i x) (+ 1 (* 2 i)))
           100 x))

(tan-cf 2 100);Value: -2.185039863261519
\end{codelisting}