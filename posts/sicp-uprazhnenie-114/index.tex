.. title: SICP Упражнение 1.14, 1.15.
.. slug: sicp-uprazhnenie-114
.. date: 2020-01-11 14:54:25 UTC+03:00
.. tags: sicp, scheme, lisp, draw
.. category: 
.. link: 
.. description: 
.. type: text

\chapter{Упражнение 1.14}

\section{Изобразите дерево иллюстрирующее процесс процедуры count-charge из секции 1.22 для 11 центов.}

\begin{codelisting}{scheme}
;; Example: Counting change
;; Сколькими способами мы можем внесни изменение в сумму
;; если у нас есть монеты 50 25 10 5 и 1 центов.
;; Есть простое решение через рекурсивную процедуру
;; если a = 0 1 способ внести изменение
;; если a < 0 0 способов
;; если n = 0, 0 способов внести изменение
(define (count-change amount)
  (cc amount 5))
(define (first-denomination kinds-of-coins)
  (cond ((= kinds-of-coins 1) 1)
        ((= kinds-of-coins 2) 5)
        ((= kinds-of-coins 3) 10)
        ((= kinds-of-coins 4) 25)
        ((= kinds-of-coins 5) 50)))

(define (cc amount kinds-of-coins)
  (cond ((= amount 0) 1)
        ((or (< amount 0) (= kinds-of-coins 0)) 0)
        (else (+ (cc amount
                     (- kinds-of-coins 1)) ;; изменение берем разные монетки
                 (cc (- amount
                        (first-denomination kinds-of-coins)) ;; изменяем сумму на значение монетки
                     kinds-of-coins)))))
\end{codelisting}

Иттерируя можно себе представить что получится что-то вроде.
\begin{codelisting}{scheme}
                            (cc 11 5)
                     (cc 11 4)     (cc -39 5)
                (cc 11 3) (cc -14 4)      0
           (cc 11 2) (cc 1 3) | 0
      (cc 11 1) (cc 6 2) | (cc 1 2) (cc -9 3)
 (cc 11 0) (cc 10 1) | (cc 6 1) (cc 1 1) | (cc 1 1) (cc -4 2) | 0
    0 | (cc 10 0) (cc 9 0) | (cc 6 0) (cc 5 1) | (cc 1 0) (cc 0 1) | (cc 1 0) (cc 0 1) | 0
          0 | 0 | 0 | (cc 5 0) (cc 4 1) | 0 | 0 | 0 | 0
          0 | 0 | (cc 4 0) (cc 3 1)
          0 | (cc 3 0) (cc 2 1)
          0 | (cc 2 0) (cc 1 1)
          0 | (cc 1 0) (cc 0 1)
          0 | 0
\end{codelisting}

\section{Каков порядок роста ($\theta(n)$) по памяти и по количеству шагов. Где n - сумма.}

По количеству шагов, давайте представим что у нас есть всего одна монетка для изменений. Тогда.

\begin{codelisting}{scheme}
        (cc amount 1)
  (cc amount 0)  (cc (amount - 1) 1)
0                0     (cc (amount - 2) 1)
\end{codelisting}

Видим что у нас левая часть всегда вырождается в 0. А порядок роста в право это просто $\theta(n)$

Теперь посмотрим что будет происходить с двумя монетками для изменения.

\begin{codelisting}{scheme}
        (cc amount 2)
  (cc amount 1)  (cc (amount - 1) 2)
\end{codelisting}

Видим что в левой части у нас в точности то что при одной монетке. А в провой части у нас каждый раз будет создаваться новая левая для одной монетки и еще одна правая. Порядок роста у нас получится $\theta(n^2)$.

И так при каждом добавлении еще одного номинала разменной монетки, порядок роста будет увеличиваться на n.

У нас в задаче 5 разменных монет, поэтому для нашей задачи порядок роста $\theta(n^5)$.

По памяти $\theta(n)$ (порядок максимального разветвления дерева).

\chapter{Упражнение 1.15}

Синус угла (указанный в радианах) может быть вычислен используя приближение что $\sin(r) = r$ если $r$ малое значение. Для итеративного уменьшения угла будем использовать тригонометрическое тождество $\sin{r} = 3\sin{r/3} - 4sin^3{r/3}$.

\begin{codelisting}{scheme}
(define (cube x) (* x x x))
(define (p x) (- (* 3 x) (* 4 (cube x))))

(define (sine angle)
   (if (not (> (abs angle) 0.1))
       angle
       (p (sine (/ angle 3.0)))))
\end{codelisting}

\section{Сколько раз выполнится процедура p применительно к sin 12.15?}

Будем делить угол на 3 пока, он не станет меньше чем 0.1. 12.15/3/3/3/3/3 = 0.049, значит 5 раз.

А если записать в общем виде это будет $\log_3{x} + a$, где а константа.

\section{Каков порядок роста этой функции.}

Порядок роста по количеству итераций равен $\theta(log{n})$.
По месту так же.