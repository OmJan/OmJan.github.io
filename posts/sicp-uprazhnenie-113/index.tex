.. title: SICP Упражнение 1.13
.. slug: sicp-uprazhnenie-113
.. date: 2020-01-10 22:44:12 UTC+03:00
.. tags: sicp, mathematics
.. category: 
.. link: 
.. description: 
.. type: text

\chapter{Упражнение 1.13}

\section{Доказать что Fib(n) равно ближайшему целому от $\phi^n/\sqrt{5}$, где $\phi = (1+\sqrt{5})/2$}

при n=0, $\phi^0/\sqrt{5} = 1/\sqrt{5}$, что округляя до целых рано 0.

при n=1, $\phi^1/\sqrt{5} = (1+\sqrt(5))/(2*\sqrt{5}) = 1/2 + 1/(2*\sqrt{5})$, что округляя до целых равно 1.
осталось доказать что Fib(n) = Fib(n-2) + Fib(n-1) (1):

$$((1+\sqrt{5})/2)^n = ((1+\sqrt{5})/2)^{n-1} + ((1+\sqrt{5})/2)^{n-2}$$, (тождество 1)
делим все на $((1+\sqrt{5})/2)^{n-2}$, получаем:
$$((1+\sqrt{5})/2)^2 = ((1+\sqrt{5})/2) + 1$$
$$(1+\sqrt{5})^2 = 2+2\sqrt{5} + 4$$
$$0 = 0$$

Ч.т.д.

\section{Доказать по индукции что функцию Фибоначи можно считать как $f(n) = (\phi^n-\psi^n)/\sqrt{5}$, где $\phi = (1+\sqrt{5})/2$ и $\psi = (1-\sqrt{5})/2$}

при n=0, $(\phi^0-\psi^0)/\sqrt{5} = 0$

при n=1, $((1+\sqrt{5})/2)-(1-\sqrt{5})/2))/\sqrt(5) = 1$

осталось доказать что Fib(n) = Fib(n-2) + Fib(n-1) (2):

$$(((1+\sqrt{5})/2)^n - ((1-\sqrt{5})/2)^n)/\sqrt{5} = (((1+\sqrt{5})/2)^{n-1} - ((1-\sqrt{5})/2)^{n-1})/\sqrt{5} + (((1+\sqrt{5})/2)^{n-2} - ((1-\sqrt{5})/2)^{n-2})/\sqrt{5}$$
$$((1+\sqrt{5})/2)^n - ((1-\sqrt{5})/2)^n = ((1+\sqrt{5})/2)^{n-1} - ((1-\sqrt{5})/2)^{n-1} + ((1+\sqrt{5})/2)^{n-2} - ((1-\sqrt{5})/2)^{n-2}$$

применяя тождество 1, получаем:

$$((1-\sqrt{5})/2)^n = ((1-\sqrt{5})/2)^{n-1} + ((1-\sqrt{5})/2)^{n-2}$$

доказывается точно также как тождество 1:

$$((1-\sqrt{5})/2)^2 = ((1-\sqrt{5})/2) + 1$$
$$(1-\sqrt{5})^2 = 2-2\sqrt{5} + 4$$
$$0 = 0$$

Ч.т.д.

